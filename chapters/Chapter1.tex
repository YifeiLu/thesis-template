\chapter{Introduction to \LaTeX}

\graphicspath{ {graphics/Chapter1/} }

\LaTeX~is a high-quality typesetting system; it includes features designed for the production of technical and scientific documentation. \LaTeX~is the de facto standard for the communication and publication of scientific documents \cite{LaTeX3Team.}.

At the very beginning of this template and short tutorial of using \LaTeX, some basic commands are shown in the following section.

\section{Basic commands and symbols}

	\paragraph{Insert quotation marks}
	
	In \LaTeX, the quotation marks are not recognized as in Microsoft Word or some other text-editing environment. If you type twice {\verb|"|}, the output will be \colorbox{yellow!50}{"some quotation"}. Hence, one should use {\verb|``|} combined with {\verb|"|} in \LaTeX~environment instead and the output will be \colorbox{yellow!50}{``some quotation"}. You can also check this in the source code.
	
	\paragraph{Insert space symbol}
	
	In \LaTeX, the ``space" used in your code will be shown as expected in most cases, however it will not be shown as a space in the generated pdf-file if typed after a \LaTeX~command. To add a space character in such cases, one should try to insert a {\color{blue}{\verb|~|}} symbol to generate an extra space character in the text. For example, \colorbox{yellow!50}{\LaTeX~is a typesetting system}, instead of \colorbox{yellow!50}{\LaTeX is a typesetting system}.
	
	\paragraph{Type special characters}
	
	There are some characters defined as special characters in \LaTeX, e.g. \%. To type such characters in the text, try to add a backslash before them ({\color{blue}{\verb|\%|}} instead of {\color{blue}{\verb|%|}}).
	
	\paragraph{Keep numbers and units in the same line}
	
	Sometimes it is necessary to add some number with units in the text, however the numbers and their units should not be divided into two different lines, hence try to use {\verb|{\,}|} instead of ``space" character in this case. You can check the source code in the corresponding  \href{chapters/Chapter1.tex}{tex-file}.
	
	For example, you may write:
	
	In 2017, the electricity generated by \acrshort{PV} was roughly 38.4 TWh, while wind about \colorbox{yellow!50}{104} \colorbox{yellow!50}{TWh} \cite{Burger.20180508}.
	
	% Original
	% In 2017, the electricity generated by \acrshort{PV} was roughly 38.4 TWh, while wind about 104 TWh.
	
	However, it maybe better to write the sentence as:
	
	In 2017, the electricity generated by \acrshort{PV} was roughly 38.4{\,}TWh, while wind about \colorbox{yellow!50}{104{\,}TWh}.
	
	% Original
	% In 2017, the electricity generated by \acrshort{PV} was roughly 38.4{\,}TWh, while wind about 104{\,}TWh.
	
	\paragraph{Font settings}
	
	All font types and sizes are already defined in the \href{thesis-template.cls}{cls-file}, if you want to change the font types, formats or sizes you can check and change the corresponding lines of code (search for keyword ``font"). 
	
	You can get a list of \LaTeX supported fonts \href{http://www.tug.dk/FontCatalogue/}{here}. To make sure this template can work on different operating systems, in the document class file only the font types directly accessible with \LaTeX~packages are used. Of course it is possible to use other fonts, e.g. Times New Roman or Tahoma on Windows platform.
	
	\paragraph{Define and use colors}
	
	\LaTeX~offers numerous predefined colors, however in certain cases it is still necessary to define new customized colors. To do so, one can use the {\verb|\definecolor|} command in \LaTeX, where the first parameter is name of the customized color, the second one is the model chose to define the color (xcolor offers gray, rgb, RGB, HTML and cmyk, you can check the corresponding definitions \href{https://en.wikibooks.org/wiki/LaTeX/Colors#Color_Models}{here}) and the third one is description of the color. You can search online for color calculators or pickers to find the corresponding values for colors. The following code defines a new color so-called ``new-color":
	
	{\color{blue}{\verb|\definecolor{new-color}{RGB}{0, 170, 165}|}}
	\definecolor{new-color}{RGB}{0, 170, 165}
	
	You may use {\color{blue}{\verb|{\color{color}text}|}} or {\color{blue}{\verb|\textcolor{color}{text}|}} to change text color. To give an example, the customized color defined before is used. With the code {\color{blue}{\verb|{\color{new-color}example}|}} it returns {\color{new-color}example}. 
	
	You can also use ! character to use percentage colors or even mix different colors to get a new one (e.g. {\color{blue}{\verb|red!50|}} is 50\% red). 
	
	For example, colors of following numbers are fading from 100\% to 10\% of \LaTeX~predefined red {\color{red!100}0}{\color{red!90}1}{\color{red!80}2}{\color{red!70}3}{\color{red!60}4}{\color{red!50}5}{\color{red!40}6}{\color{red!30}7}{\color{red!20}8}{\color{red!10}9}. This method is especially helpful to set colors for a table, which will be discussed in chapter 4.
	
	It is also feasible to get a new color by mixing defined colors. For example one can mix 50\% yellow with 90\% red {\color{blue}{\verb|{\color{yellow!50!red!90}mixed color}|}}, the output is {\color{yellow!50!red!90}mixed color}.
	
