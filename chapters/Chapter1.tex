\chapter{Introduction to \LaTeX}

\graphicspath{ {graphics/Chapter1/} }

\LaTeX~is a high-quality typesetting system; it includes features designed for the production of technical and scientific documentation. \LaTeX~is the de facto standard for the communication and publication of scientific documents \cite{LaTeX3Team.}.

At the very beginning of this template and short tutorial of using \LaTeX, some basic commands are shown in the following section.

\section{Basic commands and symbols}
	
	In \LaTeX, the quotation marks are not recognized as in Microsoft Word or some other text-editing environment. If you type twice {\verb|"|}, the output will be "some quotation". Hence, one should use {\verb|``|} combined with {\verb|"|} in \LaTeX~environment instead and the output will be ``some quotation". You can also check this in the source code.
	
	In \LaTeX, the ``space" used in your code after a common command will not be shown as a space in the generated pdf-file. To add such a space, one should try to insert a {\verb|~|} symbol to generate an extra space character in the text. For example, \LaTeX~is a typesetting system , instead of \LaTeX is a typesetting system.
	
	There are some characters defined as special characters in \LaTeX, e.g. \%. To type such characters in the text, try to add a backslash before them ({\verb|\%|} instead of {\verb|%|}).
	
	Sometimes it is necessary to add some number with units in the text, however the numbers and their units should not be divided into two different lines, hence try to use {\verb|{\,}|} instead of ``space" character in this case.
	
	For example, you may write:
	
	In 2017, the electricity generated by \acrshort{PV} was roughly 38.4 TWh, while wind about 104 TWh \cite{Burger.20180508}.
	
	However, it maybe better to write the sentence as:
	
	In 2017, the electricity generated by \acrshort{PV} was roughly 38.4{\,}TWh, while wind about 104{\,}TWh.
	
	
